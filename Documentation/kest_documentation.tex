\documentclass[12pt]{amsart}

\usepackage{amsmath} 
\usepackage{amssymb} 
\usepackage{amsthm}
\usepackage{amsbsy}
\usepackage{fullpage}
\usepackage[utf8]{inputenc}
\usepackage{algpseudocode}
\usepackage{algorithm}

%%% For commutative diagrams
\usepackage{tikz}
\usepackage[all]{xy}
\usetikzlibrary{matrix,arrows}
%%%%%%%%%%%%%%%%%%%%%%%%%%%

\usepackage[unicode=true,
colorlinks=true,
urlcolor=blue,
anchorcolor=blue,
linkcolor=blue,
citecolor=blue]{hyperref}


\newif\ifFrench
\Frenchfalse  %%%%%%%%%%%%%% Change language here

\ifFrench
\usepackage[frenchb]{babel} %Removing turns ``Demonstration'' into ``Proof''  in the proof env.
\newtheorem{theorem}{Théorème}
\newtheorem{lemma}[theorem]{Lemme}
\newtheorem{proposition}[theorem]{Proposition}
\newtheorem{corollary}[theorem]{Corollaire}
\newtheorem{theodef}[theorem]{Théorème and Définition}
\newtheorem{thesis}[theorem]{Thèse}
\theoremstyle{definition}
\newtheorem{definition}{Définition}
\newtheorem{example}{Exemple}
\theoremstyle{remark}
\newtheorem*{remark}{Remarque}
\newtheorem*{notation}{Notation}

\else
\newtheorem{theorem}{Theorem}
\newtheorem{lemma}[theorem]{Lemma}
\newtheorem{proposition}[theorem]{Proposition}
\newtheorem{corollary}[theorem]{Corollaire}
\newtheorem{theodef}[theorem]{Theoreme and Definition}
\newtheorem{thesis}[theorem]{Thesis}
\theoremstyle{definition}
\newtheorem{definition}{Definition}
\newtheorem{example}{Example}
\theoremstyle{remark}
\newtheorem*{remark}{Remark}
\newtheorem*{notation}{Notation}
\fi

\title{$K_{est}$ : A Brief Summary}
\author{}
\date{}

\begin{document}

	\maketitle

	The goal of these few pages is to give an overview of the algorithm implemented in $RecursiveCompression.lhs$ and which is at the core
	of $K_{est}$. By the time he/she reaches the end this document, the reader will hopefully be able to write the same algorithm (or a better one) in a programming 
	language of his/her choice. This should also provide an insight on why it performs the way it does compared to Lempel-Ziv \cite{LZ},
	what is the exact meaning of the parameters, etc\dots

	\section*{Preliminaries}


	\subsection*{Notations}

		In the rest of this document, $\Sigma = \{0,1\}$ is an alphabet over two symbols, here $0$ and $1$. $s \in \Sigma^n$ is a binary string
		of length $n$ and $s_i,~1 \leq i \leq n$ is the $i^{th}$ symbol of $s$. $s_i$ can equivalently be seen as a symbol ($i.e.$ an element
		of $\Sigma$) or as a string of length $1$. We will use $a \cdot b$ to denote concatenation of two strings, even if $a$, $b$ or both 
		are symbols, thus abstracting away from the more constrictive framework of type theory. $length : \Sigma^\ast \to \mathbb{N}$ is
		the usual length function and $\preceq$ is a total order relation on
		$\Sigma^\ast$ defined by
		\[ s_1 \preceq s_2 \Leftrightarrow length(a) \leq length(b).\]
		
		Moreover, recall that there are ${n \choose k}$ binary strings $s \in \Sigma^n$ with $k$ ones. The bijections between
		the set of such strings and $\{1\dots {n \choose k}\}$ are called \emph{ranking} functions. The inverse operation is called
		\emph{unranking} and both can be implemented in an efficient way. All ranking functions are identical up to 
		isomorphism and, since it does not matter which one we choose, from now on \texttt{rank(s)} will designate some ranking function, fixed once and for all.

		Finally, $s_{i:j}$ is the shorthand notation
		for the string $s_i \cdot s_{i+1} \cdot \dots \cdot s_{j}$ whenever $1 \leq i \leq j \leq n$.

		\subsection*{A few definitions}

			The following section makes use of a few simple functions that are not directly related to the algorithm and are defined here. 

			\begin{itemize}
				\item \texttt{selfEncode} : Because we aim at estimating the prefix free complexity, strings must be self-delimited. This is a achieved using a
					fairly naive method. If $s$ is a string of length $n$, then call $l$ the binary representation of its length. Then, encode $l$, of length
					$m$ using $2m$ bits by prepending a string of length $m$ of the form $00 \dots 0 \cdot 1$ to $l$. $0$ can be interpreted as ``keep reading''
					and $1$ as ``stop''. This tells the decoder that the length of the message is encoded in the $m$ following bits. For the pseudo-code see Function 1 
				\item \texttt{encodeBinary} : As stated above, any string of length $n$ with $k$ ones can be ranked. Once one knows $n, k$ and the rank then the string
					can be uniquely identified. This is what the encodeBinary function does. It encodes the number of ones or zeros (depending on which one is smaller) and the rank of the string. See Function 2 for the pseudo-code.
				\item \texttt{encodeNAry} : This function takes a string and a pattern (a substring of $s$) and replaces all the occurrences of $p$ in $s$ by ones. All the
					other symbols are replaced by zeros. The pattern is encoded in a self-delimited fashion and prepended to the encoding of the string with the patterns cut out, as computed by \texttt{encodeBinary}. See Function 3 for the pseudo-code.
			\end{itemize}

			\begin{algorithm}{Function 1 : selfEncode}
			\begin{algorithmic}
				\Function{selfEncode}{$s$}
					\State $ e \gets \epsilon$ \Comment{An empty string}
					\State $ l \gets $ BIN($length(s) - 1$) \Comment{BIN is a function sending a number in base $10$ to its representation in base $2$}. 
					\State $ e \gets e \cdot ((length(l) - 1) \times 0) \cdot 1 \cdot l \cdot s$
					\State \textbf{Return} $e$
				\EndFunction
			\end{algorithmic}
		\end{algorithm}
	
		\begin{algorithm}{Function 2 : encodeBinary}
			\begin{algorithmic}
				\Function{encodeBinary}{$s$}
					\State $e \gets \epsilon$ \Comment{An empty string}
					\State nZeros $\gets $ number of zeros in $s$
					\State nOnes $\gets$ number of ones in $s$
					\If{ nZeros $\leq$ nOnes}
						\State $e \gets e \cdot 0$
						\State $e \gets e \cdot $ selfEncode(BIN(nZeros) ) \Comment{Where BIN is a function sending a number in base $10$ to its representation in base $2$}
					\Else
						\State $e \gets e \cdot 1$
						\State $e \gets e \cdot $ selfEncode(BIN(nOnes) )
					\EndIf
					\State $e \gets e \cdot $ rank(s)
					\State \textbf{Return} $e$
				\EndFunction
			\end{algorithmic}
		\end{algorithm}

		\begin{algorithm}{Function 3 : encodeNAry}
			\begin{algorithmic}
				\Function{encodeNAry}{$b,p,s$} \Comment{$b$ is a boolean,$p$ is string, the ``pattern'' and $s$ is the original string.}
					\State $e \gets \epsilon$ \Comment{An empty string}
					\If{$b$ is True}
						\State $e \gets e \cdot 1$
					\Else
						\State $e \gets e \cdot 0$
					\EndIf
					\State $e \gets e \cdot $ selfEncode($p$) \Comment{Encode the pattern in a self-delimited way.}
					\State $s' \gets $ $s$ in which all occurrences of $p$ are replaced by $1$ and all the other symbols are replaced by $0$. 
					\State $e \gets e \cdot $ encodeBinary($s'$)
					\State \textbf{Return} $e$
				\EndFunction
			\end{algorithmic}
		\end{algorithm}
	

		\section*{Algorithm}

			This algorithm estimates the Kolmogorov complexity by searching for patterns in strings. This is done recursively, until no interesting pattern
			can be detected. This paragraph will focus on an intuitive description of the algorithm. For the pseudo-code see Algorithm \ref{algo}.

			Here is the work flow :
			\begin{enumerate}
				\item Take a string $s$ of length $n$.
				\item For a given value of $t \leq n$, enumerate all substrings of length $t$ of $s$. There are $n-t$ such strings. We will call such substrings
					\emph{patterns}.
				\item For a given pattern $p$, replace all occurrences of $p$ in $s$ by ones and all the other symbols by zeros. This gives a simpler binary string that can be encoded using the encodeBinary function. 
				\item Cut out all the occurrences of $p$ in $s$. This gives a shorter string $s'$ that can itself be encoded using the same algorithm.
				\item Do this for all patterns $p$ and all allowed values of $t$.
				\item Return the shortest encoding.
			\end{enumerate}

		\begin{algorithm}
			\caption{An algorithm to estimate the prefix-free Kolmogorov complexity of a binary string.}
			\label{algo}
			\begin{algorithmic}
				\Procedure{encode}{$s,mt$}\Comment{Encode a string $s$, looking for patterns of length $\leq mt$}
					\State encodedStrings = \{\} \Comment{An empty list}
					\For {$t=0 \to mt$}
						\State $e \gets$ encodeT($s,t$) 
						\State Add $e$ to encodedStrings
					\EndFor
					\State \textbf{return} min(encodedStrings) \Comment{Return the shortest string according to $\preceq$.}
				\EndProcedure

				\Function{encodeT}{$s,t$}
				\State patterns $\gets$ the list $\{s_{1:t}, s_{2:(t+1)}, \dots , s_{(n-t):n} \}$.
				\State $best \gets 0 \times 2000$ \Comment{A very long string}
				\For {$p$ in patterns} \Comment{First half : Detect only one pattern.}
					\If {$t$ is $1$} \Comment(The pattern is not really a pattern, it is just one symbol)
						\State $e \gets 1 \cdot $ selfDelimited($p$) $ \cdot $ encodeBinary($s$)
						\State If $e \preceq best$ then $best \gets e$. 
						\Else \Comment{This is a proper pattern of length $2$ or more.}
							\State $e \gets $ EncodeNAry(True,$p$,$s$)
							\State $s' \gets $ $s$ with all the occurrences of $p$ replaced by the empty string.
							\If {$s'$ is $\epsilon$}
								\State If $e \preceq best$ then $best \gets e$. 
							\Else
								\State $e \gets e \cdot $ encodeBinary($s'$)
								\State If $e \preceq best$ then $best \gets e$. 
							\EndIf
					\EndIf
				\EndFor
				\For {$p$ in patterns}
					\State $e \gets $ encodeNAry(False,$p$,$s$)
					\State $s' \gets $ $s$ with all the occurrences of $p$ replaced by the empty string.
					\If {$s'$ is $\epsilon$}
					\State \textbf{Continue}
					\Else
					\For {$i = 1 \to min(t,length(s'))$} 
						\State $e' = e \cdot $ encodeT($s',i$)
						\State If $e' \preceq best$ then $best \gets e'$. 
					\EndFor
					\EndIf
				\EndFor
				\textbf{Return} $best$
				\EndFunction
			\end{algorithmic}
		\end{algorithm}

	\bibliographystyle{alpha}
	\bibliography{bibliography}

\end{document}
